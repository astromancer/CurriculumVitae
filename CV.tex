% LaTeX resume using res.cls
%& -job-name=newfilenameialwayswanted
% \listfiles
\documentclass{article}

% \usepackage[urw-garamond]{mathdesign}
\usepackage[T1]{fontenc}
\usepackage[utf8]{inputenc}
% \usepackage{kpfonts} % lmodern
\usepackage{bold-extra}  % It works by building a bold Computer Modern small-caps font on the fly and including it for you
% \normalfont %to load T1lmr.fd 
% \DeclareFontShape{T1}{lmr}{bx}{sc} { <-> ssub * cmr/bx/sc }{}
\usepackage[pdftex]{hyperref}    % embedding hyperlinks
\hypersetup{
     colorlinks=true,
     linkcolor=black,
    filecolor=blue,
    urlcolor=blue,
}

\usepackage[depth=3]{bookmark}    % pdf bookmarks (document outline)
% \setcounter{secnumdepth}{3}
\makeatletter
\renewcommand\@seccntformat[1]{}  % suppress section numbering
\makeatother

%\usepackage{amssymb}
% \usepackage{helvetica}     % uses helvetica postscript font (download helvetica.sty)
%\usepackage{newcent}        % uses new century schoolbook postscript font
\usepackage{xspace}
\usepackage{tabularx}
\usepackage{fancyhdr}          % use fancyhdr package to get 2-line header
\usepackage[acronym]{glossaries}    % manage acronyms
\usepackage{titlesec}        % edit section titles
\usepackage{soulutf8}        % underlining
\usepackage{xparse}        % better macros
\usepackage{xifthen}        % conditionals for custom macros
\usepackage{xstring}        % string ops
\usepackage[object=vectorian]{pgfornament}   % Aesthetics
\usepackage{graphbox}         % graphics tools.  loads graphicx package


%%%%%%%%%%%%%%%%%%%%%%%% layout %%%%%%%%%%%%%%%%%%%%%%%%%
\usepackage[margin=0.75in]{geometry}
\setlength{\topmargin}{-0.6in}  % Start text higher on the page 
\setlength{\textheight}{9.5in}  % increase textheight to fit more on a page
\setlength{\headsep}{0.2in}     % space between header and text
\setlength{\headheight}{12pt}   % make room for header
\setlength{\parindent}{0cm}

% Increase table vertical space
\renewcommand{\arraystretch}{1.35}

%%%%%%%%%%%%%%%%%%%%%%%% Page headers %%%%%%%%%%%%%%%%%%%%%%%%%
%\renewcommand{\headrulewidth}{0pt} % suppress line drawn by default by fancyhdr
\lhead{Hannes Breytenbach} % \hspace*{-\sectionwidth} % force lhead all the way left
\rhead{Page \thepage}  % put page number at right
\cfoot{}  % the footer is empty
\pagestyle{fancy} % set pagestyle for the document

%%%%%%%%%%%%%%%%%%%%%%%% section headers %%%%%%%%%%%%%%%%%%%%%%%%%

% from the titlesec package
%\titleformat{ command }
%             [ shape ]
%             { format }{ label }{ sep }{ before-code }[ after-code ]
% \section

\titleformat{\section}[block]
  {\filcenter\Large\bfseries\scshape}%% \normalfont since CMR doesn't have bold small caps. If you use some other font like libertine then you can have bold small caps
  {}
  {0pt}
  {\\}[\vspace{2pt}\titlerule]

\titleformat{\subsection}
  {\normalfont\Large\bfseries}%
  {}
  {1em}
  {}


% \newcommand{\sectionfont}{\Large\bfseries\scshape}
% 
% \titleformat{\section}[block]
%     {\titlerule
%      \vspace{0.5ex}%
%      \sectionfont}
%     {\thesection}{1em}
%     {\sectionfont}[\titlerule]
    
% \titleformat{name=\section,numberless}
% {\filcenter\Large\bfseries}
% {}
% {.5em}
% {}

% Custom section headers
% TODO: the section headers are now not registering in the pdf, so can't jump around with links.  FIXME
% \renewcommand{\section}[1]{%
%   \vspace{0.4cm}%
%   \begin{table}[!htp]%          % not h on its own
%     \newcolumntype{C}{>{\centering\arraybackslash}X}%  % centering
%     \setlength\extrarowheight{3pt}%       % extra padding
%     \noindent%             % otherwise the line will be too wide by \parindent
%     \begin{tabularx}{\textwidth}{C}%
%       \hline \hline%
%   \large \textbf{\textsc{#1}} \\%
%       \hline \hline%
%     \end{tabularx}
%   \end{table}
% }

%%%%%%%%%%%%%%%%%%%% Bibliography (Publications) %%%%%%%%%%%%%%%%%%%%
\usepackage{extern/aas_macros}
\usepackage[
  backend=biber,
  style=trad-abbrv,%numeric
  sorting=ydnt, %none, % sort by date
  doi=false,
  url=false,
  isbn=false,
  eprint=false,
  maxbibnames=99
]{biblatex}

% add style file (abbreviations)
% \bibliographystyle{aasjournal} # NOT WITH BIBLATEX

% ---------------------------------------------------------------------------- %
% Better formatting for line breaks in bibliography
% https://tex.stackexchange.com/a/442317
% \appto\bibfont{}
\setlength{\emergencystretch}{1em}

% Line breaks encouraged before page ranges
\protected\long\def\blx@mkpageprefix#1[#2]#3{%
  \ifstrequal{#1}{page}
    {}
    {\ifnumeral{#3}
      {\bibstring{#1}\ppspace}
      {\ifnumerals{#3}
         {\bibstring{#1s}\ppspace}
         {\def\pno{\bibstring{#1}}%
          \def\ppno{\bibstring{#1s}}}}}%
  \blx@mkpageprefix@i[#2]{#3}}

% ---------------------------------------------------------------------------- %
% remove comma between author and year in citations
% \renewcommand*{\nameyeardelim}{\addspace}

%! Remove "In:" (for journal pointers) from bibliography entries 
\renewbibmacro{in:}{}

% Add line space between bibliography entries
% \setlength{\bibitemsep}{\baselineskip}

% Don't print `eid' if same as `pages' field
% https://tex.stackexchange.com/a/297126/102109
\AtEveryBibitem{\iffieldsequal{eid}{pages}{\clearfield{pages}}{}}

% Don't print url, eprint if doi present
\DeclareSourcemap{
  \maps[datatype=bibtex]{
    \map[overwrite]{
      \step[fieldsource=doi, final]
      \step[fieldset=url, null]
      \step[fieldset=eprint, null]
    }  
  }
}
% ---------------------------------------------------------------------------- %
% Hyperlink bib entry titles
% https://tex.stackexchange.com/a/48409/102109

\newbibmacro{format_bib_title}[1]{%
  \iffieldundef{doi}{%
    \iffieldundef{url}{%
      % \iffieldundef{adsurl}{%
        \iffieldundef{isbn}{%
          \iffieldundef{issn}{%
            #1%
          }{%
            \href{https://openlibrary.org/search?issn=\thefield{issn}}{#1}%
          }%
        }{%
          \href{https://openlibrary.org/search?isbn=\thefield{isbn}}{#1}%
        }%
      % }{%
      %   \href{\thefield{url}}{#1}%
      % }%
    }{%
        \href{\thefield{url}}{#1}%
    }%
    }{%
    \href{http://dx.doi.org/\thefield{doi}}{#1}%
  }%
}

\DeclareFieldFormat{title}{\usebibmacro{format_bib_title}{\mkbibemph{#1}}}
\DeclareFieldFormat[article,incollection,inproceedings]{title}%
    {\usebibmacro{format_bib_title}{\mkbibemph{#1}}}

\DeclareFieldFormat{date}{\bf #1}

% \renewbibmacro{by:}{}
  
% ---------------------------------------------------------------------------- %
% Add content
\addbibresource{publications/articles.bib}
\addbibresource{publications/proceedings.bib}
\addbibresource{publications/shortpub.bib}
% subsubbibliography
\defbibheading{subsubbibliography}[\refname]{\subsubsection*{#1}}
      
% also some weirdly translated html characters from ADS end up in the bibtex entries which cause errors
\newcommand{\rsquo}{'} 

% Make bib entry titles italic
% https://tex.stackexchange.com/questions/311816/want-title-in-simple-numeric-not-italic-through-bibliography
\DeclareFieldFormat{title}{\emph{#1}}

% Print article title first
% see: https://tex.stackexchange.com/questions/297087/putting-the-title-first-in-the-bibliography
% \DeclareBibliographyCategory{mine}
% \addtocategory{mine}{companion,sigfridsson}
% \nocite{companion,sigfridsson}

\newcommand{\nameuse}[1]{%
  \def\do##1{\settoggle{blx@use##1}{#1}}%
  \dolistcsloop{blx@datamodel@names}}

\newcommand{\nameusesave}{%
  \def\do##1{%
    \providetoggle{blx@save@use##1}%
    \iftoggle{blx@use##1}{\toggletrue{blx@save@use##1}}{\togglefalse{blx@save@use##1}}%
  }%
  \dolistcsloop{blx@datamodel@names}}

\newcommand{\nameuserestore}{%
  \def\do##1{%
    \iftoggle{blx@save@use##1}{\toggletrue{blx@use##1}}{\togglefalse{blx@use##1}}%
  }%
  \dolistcsloop{blx@datamodel@names}}


%%%%%%%%%%%%%%%%%%%%%%%%%%%%%%%%%%%%%%%%%%%%%%%%%%%%%%%%%%%%%%%%%%%%%

 %%%%%%%%%%%%%%%%%%%%%%%% my custom commands %%%%%%%%%%%%%%%%%%%%%%%%%

% public HTML location of presentations / posters
\newcommand{\homeURL}{http://www.saao.ac.za/~hannes}
\newcommand{\pressURL}{\homeURL/presentations}
\newcommand{\talksURL}{\pressURL/talks}
\newcommand{\postersURL}{\pressURL/posters}
\newcommand{\hacksURL}{\pressURL/hacks}
\newcommand{\thesesURL}{\homeURL/theses}

% item points
\newcommand{\itm}[1]{\textbf{#1}}

% sub-item bullet points
\newcommand{\tb}{\textbullet}
  
% table bullet points in multicolumn
\newcommand{\tblblt}[2]{%
  \multicolumn{#1}{l}{\hspace{10pt}\tb\hspace{10pt}\parbox{0.9\textwidth}{#2}}% \vspace{5pt}%
  }

% linking commands
\NewDocumentCommand \subItem { m m o O{} }
  % Arguments are:   #1: Item id text
  %      #2: Description    
  %      #3 Optional url
  %      #4 Optional path to prepend to url
  {  
    \IfValueTF {#3}
      {#1: \href{#4#3}{``#2''}}
      {#1: ``#2''}
  }

\NewDocumentCommand \subItemHelper { m m m m }
  { 
    \ifthenelse{\isempty{#3}}%
      {\subItem{#1}{#2}}%         if link URL is empty
      {\subItem{#1}{#2}[#3][#4]}%     if link URL given
  }

\DeclareExpandableDocumentCommand \subItemTbl { m m o m } {%
  \tblblt{3}{\subItemHelper{#1}{#2}{#3}{#4}}%
  }
  
\DeclareExpandableDocumentCommand \talk { O{Talk} m m } {%
  \subItemTbl{#1}{#2}[#3]{\talksURL/}
  }

\DeclareExpandableDocumentCommand \poster { O{Poster} m m } {%
  \subItemTbl{#1}{#2}[#3]{\postersURL/}
  }

 \DeclareExpandableDocumentCommand \hack { O{Hack} m m } {%
  \subItemTbl{#1}{#2}[#3]{\hacksURL/}
  }


% Create condensed URLs
% https://tex.stackexchange.com/questions/153215/how-to-do-multiple-string-replacements
\newcommand{\urlescapestep}[2]{%
  \expandafter\StrSubstitute\expandafter{\x}{#1}{#2}[\x]%
}
\newcommand{\urlcondense}[1]{{% 
  \noexpandarg % suppress expansions made by xstring
  \StrSubstitute{#1}{https://}{}[\x]% first step
  \urlescapestep{http://}{}%
  \urlescapestep{www.}{}%
  \x}}

% Create link to online profile url + linked logo
\newcommand{\onlineProfile}[2]{%
  \href{#2}{\includegraphics[align=c, width=5mm]{logos/#1}} \href{#2}{\urlcondense{#2}}%
  }
  
% Skills
\newcommand{\skill}[2]{%
  \hspace{10pt}\tb\hspace{10pt} \emph{#1} & #2          \\
  }
  
% Add a referee
\newcommand{\reference}[5]{%
  %  #1 : name, #2 position, #3 affiliation, #4 email, #5 phone
  \multicolumn{2}{l}{\parbox{0.95\textwidth}{\textbf{#1}}} \\
  & #2, #3  \\
  & #4      \\
  & #5          \\
 }
  
  
%%%%%%%%%%%%%%%%%%%%%%%% acronyms %%%%%%%%%%%%%%%%%%%%%%%%%
\makeglossaries
\newacronym{uct}{UCT}{University of Cape Town}
\newcommand{\uct}{\gls*{uct}\xspace}

\newacronym{uj}{UJ}{University of Johannesburg}
\newcommand{\uj}{\acrshort*{uj}\xspace}

\newacronym{up}{UP}{University of Pretoria}

\newacronym{wits}{WITS}{University of the Witwatersrand}
\newcommand{\wits}{\acrshort*{wits}\xspace}

\newacronym{saao}{SAAO}{South African Astronomical Observatory}
\newcommand{\saao}{\gls*{saao}\xspace}

\newacronym{ska}{SKA}{Square Kilometre Array}
\newcommand{\ska}{\gls*{ska}\xspace}

\newacronym{narit}{NARIT}{National Astronomical Research Institute of Thailand}
\newcommand{\narit}{\gls*{narit}\xspace}

\newacronym{pdp}{PDP}{Postgraduate Development Programme}
\newcommand{\pdp}{\gls*{pdp}\xspace}

\newacronym{saip}{SAIP}{South African Institute of Physics}
\newcommand{\saip}{\gls*{saip}\xspace}

\newacronym{hartRAO}{HartRAO}{Hartebeest Hoek Radio Astronomy Observatory}
\newcommand{\hartRAO}{\acrshort*{hartRAO}\xspace}

\newacronym{stias}{STIAS}{Stellenbosch Institute for Advanced Studies}
\newcommand{\stias}{\acrshort*{stias}\xspace}

\newacronym{salt}{SALT}{South African Large Telescope}
\newcommand{\salt}{\acrshort*{salt}\xspace}

\newacronym{shoc}{SHOC}{Sutherland High-speed Optical Camera}
\newcommand{\shoc}{\gls*{shoc}\xspace}

\newacronym{chpc}{CHPC}{Centre for High Performance Computing}
\newacronym{nassp}{NASSP}{National Astrophysics and Space Science Program}

\newacronym{msc}{MSC}{Mountain and Ski Club}
\newcommand{\msc}{\gls*{msc}\xspace}

\newacronym{mcsa}{MCSA}{Mountain Club of South Africa}
\newcommand{\mcsa}{\gls*{mcsa}\xspace}

% \newacronym{mCVs}{mCV}{Magnetic Cataclysmic Variable}


% Content starts here

\begin{document} 
\thispagestyle{empty} % this page does not have a header

%  HEADING
\begin{center}
  \emph{Curriculum Vitae}\\
  \vspace{0.4cm}
%   \rput[r](-3pt,3pt){\pgfornament[scale=.35]{72}}%
  \huge \textbf{\textsc{Hannes Breytenbach}}\\
%   \rput[l](3pt,3pt){\pgfornament[scale=.35]{73}}\\
  \vspace{0.2cm}
  \normalsize
  \emph{\large hannes@saao.ac.za}\\
  \emph{\large +27 78 192 2193 } \\
  % \emph{\tb\ Room 5-42, RW James Bldg., Dept. of Astronomy, University of Cape Town \tb}\\ 
%   \emph{}\\
  \vspace*{\baselineskip}
  \pgfornament[width = 7cm,
               color = black]{88}
   \vspace*{-\baselineskip}

  \end{center}

\vspace{-\baselineskip}



\section{About Me}
% ---------------------------------------------------------------------------- %

I'm a published researcher in the field of astrophysics with a passion for data
science and data visualization. I am currently completing my MSc in astrophysics
at \uct and expect to graduate in December 2024.\\
%I have led a number of successful science proposals employing the
%\acrlong*{salt} (amongst others) to study the rapid variability of an enigmatic
%class of binary stars known as magnetic Cataclysmic Variables, and to date have
%(co-)authored 10 peer-reviewed scientific publications. %
%


I'm a proficient full stack {\tt python} developer with 12 years of coding
experience and contributions to a number of open-source scientific computing and
visualization libraries. I am passionate about open source and have authored and
maintain a number of software libraries for applications in astronomy. I have
extensive experience in statistical model building and using machine learning to
gain valuable insights into data. I have participated in, and later led
independent research into developing machine learning models for predicting the
onset of seizures in patients suffering from epilepsy using EEG data.\\


% I also have a keen interest in applying Bayesian reasoning to scientific applications
%and I am currently developing a framework for the analysis of astronomical
%video data in order to study the physics of accreting plasmas. 
%
%I have experience in cluster computing, having made extensive use of the SAAO
%Mons/Mensa clusters for data analysis during my PhD.


My main hobbies are outdoor adventure sports such as rock-climbing and
high-lining, and I have led- and participated in numerous international
expeditions to remote mountainous regions on earth.




\section{Personal Info}
% % ---------------------------------------------------------------------------- %

\begin{tabular}[h!]{l l} 
  \textbf{Full Names:}          & Johannes Benjamin Breytenbach\\
  \textbf{Current Occupation:}  & MSc. Student at \saao \& \\ & \uct\\
  \textbf{Work Address:}        & Room 5-42, RW James Bldg., Dept. of Astronomy, 
                                  University of Cape Town \\
  \textbf{Date of Birth:}       & 26 June 1987  \\
%   \textbf{Place of Birth:}  & Santiago, Chile\\
  \textbf{Citizenship:}         & South African\\
  \textbf{Languages:}           & English, Afrikaans (Native); German (Conversational) \\
  \textbf{Drivers License:}     & A, B

%   \textbf{Work Address:}  & Room 4-31, RW James Building \\&9 University Avenue \\&Upper Campus, \uct %\\&Woolsack Drive 
%           \\&Rondebosch, 7001 \\&Cape Town \\&South Africa
\end{tabular}




\section{Online Presence}
% ---------------------------------------------------------------------------- %

\begin{tabular}{p{0.5\textwidth} l}
  \onlineProfile{github.png}{https://github.com/astromancer}
    & \onlineProfile{Linkedin-logo.png}{https://www.linkedin.com/in/hannes-breytenbach}         \\
  \onlineProfile{RG-logo.png}{https://www.researchgate.net/profile/Hannes\_Breytenbach}
    & \onlineProfile{mendeley-logo.png}{https://www.mendeley.com/profiles/hannes-breytenbach2/} \\
  \onlineProfile{kaggle-logo.png}{https://www.kaggle.com/apodemus}
    & \onlineProfile{ORCiD-logo.png}{https://orcid.org/0000-0001-5391-2386}                     \\
  \onlineProfile{StackOverflow-logo.png}{https://stackoverflow.com/users/1098683/astromancer}  
    & \onlineProfile{fb-logo.png}{https://www.facebook.com/hannes.breytenbach.3}                \\
  % TODO personal website ??? 
\end{tabular}

\newpage



\section{Education}
% ---------------------------------------------------------------------------- %
\begin{longtable}{lll}
  \itm{2011 -- 2024}   & MSc (Astrophysics) -- \uct, \saao\ (downgraded from PhD, completed part-time) \\
    \multicolumn{3}{c}{\parbox{0.95\textwidth}{
      \begin{tabular}{p{0.175\linewidth}p{0.75\linewidth}}
        \tb\ Thesis title:   & 
          \href{\thesesURL/msc/main.pdf}
              {\emph{``A Study of Quasi-Periodic Oscillations in
                       magnetic Cataclysmic Variable Stars''}} \\
        \tb\ Supervisors:    & Dr. David Buckley, A. Prof Patrick Woudt \\
        \tb\ Modules:        & 
          Cataclysmic Variable Stars, Stellar Structures, Advanced General
          Relativity, Hot Topics in Cosmology, High Energy Astrophysics \\ 
      \end{tabular} }}
  \\ \\

  \itm{2010}     & BSc Honours (Astrophysics and Space Science) -- \uct\ \\
    \multicolumn{3}{c}{\parbox{0.95\textwidth}{
      \begin{tabular}{p{0.175\linewidth}p{0.75\linewidth}}
        \tb\ Thesis title:  & 
          \href{\thesesURL/2010NASSPHons_Sferics.pdf}
                {\emph{``The Sferic Count Rate from SANAE-IV, Antarctica''}} \\
        \tb\ Supervisors:   & Dr. Andrew Collier  \\
        \tb\ Modules:       &
          General Astrophysics, Electrodynamics, General Relativity, Computational
          Astrophysics, Galaxies and Large Scale Structure, Observational
          Techniques, Radio Astronomy\\ 
      \end{tabular} }}
  \\ \\

  \itm{2006 -- 2009}     & BSc (Physics and Astronomy) -- \gls*{up}\\ %\ae\
    \multicolumn{3}{c}{\parbox{0.95\textwidth}{
      \begin{tabular}{p{0.175\linewidth}p{0.75\linewidth}}
        \tb\ Project title:  & 
          \href{\thesesURL/2009BSc_Al100.pdf}
               {\emph{``Rutherford Backscattering Spectroscopy and X-ray Diffraction
                        Spectroscopy of Aluminium 100''}} \\
        \tb\ Supervisors:    & Prof. Chris Theron\\
        \tb\ Modules:        & 
          Quantum Mechanics, Solid State Physics, Statistical Mechanics,
          Differential Calculus, Vector Calculus, Partial Differential Equations,
          Abstract Algebra, Mathematical Modelling\\

      \end{tabular} 
  }}
  \\
\end{longtable}
% 



\section{Work \& Teaching Experience}
% ---------------------------------------------------------------------------- %
% \vspace{0.2cm}
\begin{tabular}{l p{0.55\linewidth} l}
  \itm{2021 -- 2022}  & Python Developer / Data Scientist & Float Capital \\
  \tblblt{3}{Worked on the visualization platform for diagnostics of market system behaviour.} \\
  \tblblt{3}{Researched Agent Based Modelling approach for market prediction.} \\

  \itm{2022}          &  Tutor      & \uct                \\
  \tblblt{3}{Masters course: Cataclysmic Variable stars, Practical component}  \\
  \itm{2009 -- 2015}  &  Tutor      & \uct, UP              \\
    \tblblt{3}{Tutored various subjects: Biological Physics (1st year),
     Astronomy (2nd year), Electrodynamics (Hons.)}   \\
  
  \itm{2007 -- 2008}  &  Junior Data Analyst   & \hartRAO                \\
    \tblblt{3}{Analysed radio data from astronomical masers to search for variability.}          \\
  
\end{tabular}



\section{Special Skills}
% ---------------------------------------------------------------------------- %

\subsection*{Software Development}


\begin{tabular}{l p{\dimexpr 0.75\linewidth-2\tabcolsep}}
  
  \emph{\large Scientific Computing}
  \\
  \multicolumn{2}{c}{\parbox{0.95\textwidth}{
    \begin{tabular}{p{0.15\linewidth}p{0.75\linewidth}}
      \tb \ {\bf 2012-2024}
      & Full stack python developer with 12 years of experience, specialising in
        scientific applications.   \\
      & {\bf Tools:} {\tt git, vscode, sourcery, pylance, pytest, tox} \\

      \tb \ {\bf 2018-2024}
        & Contributor to {\tt matplotlib, astropy, pymultinest} \\
        & {\bf Tools:} {\tt github, circleCI, AWS pipelines} \\

      \tb \ {\bf 2019-2024}
        & Author and maintainer of several open source packages:
          \href{https://github.com/astromancer}{my github}. \\
        & \ \ \href{https://github.com/astromancer/pyshoc}{\tt pyshoc}:
            A library for analysing data from the \shoc instrument.  \\
        & \href{https://github.com/astromancer/salticam}{\tt salticam}:
          An improved photometry pipeline for SALTICAM ultra-fast imaging camera
          on SALT. \\
        % TODO: link to paper?
      \end{tabular}
  }}
  \\

  \emph{\large Cryptocurrency}
  \\
  \multicolumn{2}{c}{\parbox{0.95\textwidth}{
    \begin{tabular}{p{0.15\linewidth}p{0.75\linewidth}}
      \tb \ {\bf 2021-2022}
      & Worked on the visualization code for diagnostics of market system 
      behaviour for the cryptocurrency trading platform Float Capital. \\
      & {\bf Tools:} {\tt plotly} \\


      \end{tabular}
  }}
  \\
  % 
  % \skill{Knowledgeable:}{git, C, R, MATLAB, Mathematica, Maple, LabView, \LaTeX, IRAF}
  
\end{tabular}

    

\subsection*{Data Science}
\begin{tabular}{l l}
    
  \emph{\large Machine Learning} \vspace{2pt}
  \\
  \multicolumn{2}{c}{\parbox{0.95\textwidth}{
    \begin{tabular}{p{0.15\linewidth}p{0.75\linewidth}}
      \tb \ {\bf 2018 -- 2024}
      & Developed \href{https://github.com/astromancer/obstools}{\tt obstools}: A
        library for automated image alignment of overlapping astronomical images etc. \\
      & {\bf Skills:} Image Segmentation, Image Registration, Optimization, Clustering \\

      \tb \ {\bf 2016}
      & Led a team in
        \href{https://www.kaggle.com/c/melbourne-university-seizure-prediction}
              {Melbourne University AES/MathWorks/NIH Seizure Prediction Challenge}:
        Model ranked top 23\% on Kaggle.\\
      & {\bf Skills:} Data Mining, Feature Engineering, Model Selection, Team Management      \\
      
      \tb \ {\bf 2014} 
      & Developed \href{https://gitlab.com/seizures_2016/eeg}{\tt eeg}:
        Algorithms for classification of EEG data for 
        \href{https://www.kaggle.com/c/seizure-prediction}
              {American Epilepsy Society Seizure Prediction Challenge}:
        Model ranked top 15\% on Kaggle.     \\
        & {\bf Skills:} Time Series Analysis, Digital Signal Processing, 
                    Spectral Estimation, Statistical Modelling, Prediction, 
                    Classification       \\
        & {\bf Tools:} {\tt scikit-learn, tensorflow, openCV, xdg-boost}      %TODO gluon, 
      \\
    \end{tabular}
  }}
  \\

  \emph{\large Signal Processing} \vspace{2pt}
  \\
  \multicolumn{2}{c}{\parbox{0.95\textwidth}{
    \begin{tabular}{p{0.15\linewidth}p{0.75\linewidth}}
      \tb \ {\bf 2020 -- 2024}
        & {Developed  {\tt \href{https://github.com/astromancer/tsa}{tsa}}: 
           Time series analysis \& spectral estimation tools to
           search for and characterize Quasi-Periodic Oscillations in magnetic
           Cataclysmic Variable stars.} \\
        & {\bf Tools:} {\tt scipy, astropy, pandas, sktime}  \\
    \end{tabular}
  }}
  \\

  \emph{\large Statistical Modelling} 
  \\
  \multicolumn{2}{c}{\parbox{0.95\textwidth}{
    \begin{tabular}{p{0.15\linewidth}p{0.75\linewidth}}
      \tb \ {\bf 2020 -- 2022}
      & {Developed and tested a generative model for (EM)CCD data
         for calibrating fast-frame rate video observations of mCVs. } \\
      & {\bf Tools:} {\tt pymultinest, emcee, lmfit}                  \\
    %   %TODO learn to use more Bayesian tools
    \end{tabular}
  }}
  \\
  
  \emph{\large Cluster Computing}
  \\
  \multicolumn{2}{c}{\parbox{0.95\textwidth}{
    \begin{tabular}{p{0.15\linewidth}p{0.75\linewidth}}
      \tb \ {\bf 2018}
        & Deployed {\tt pyshoc} on \saao Mensa cluster.    \\
        & Tools: {\tt multiprocessing, joblib}                \\
    \end{tabular}
  }}


\end{tabular}

  
\subsection*{Astronomy \& Astrophysics}
\begin{tabular}{l p{\dimexpr 0.75\linewidth-2\tabcolsep}}

  \emph{\large Observing}
  \\
  \multicolumn{2}{c}{\parbox{0.95\textwidth}{
    \begin{tabular}{p{0.15\linewidth}p{0.75\linewidth}}
      \tb \ {\bf 2012-2021}
      & Total of 135 nights of observing on \saao 1.0m, 1.9m, IRSF 
        telescopes, including remote operation. \\
      & {\bf Tools:} {\tt Aladin, ds9, js9} \\

      \tb \ {\bf 2017}
      & 7 nights operator training on SALT. \\
    \end{tabular}
  }}
  \\
  \\

  \emph{\large Proposal Writing}
  \\
  \multicolumn{2}{c}{\parbox{0.95\textwidth}{
    \begin{tabular}{p{0.15\linewidth}p{0.75\linewidth}}
      \tb \ {\bf 2014-2015} 
      & PI of 2 successful SALT science proposals.  \\
      \tb \ {\bf 2012-2018} 
      & Co-I of several observing proposals for SAAO 1.9m, 1.0m telescopes, as
        well as for the 
        \href{http://www.narit.or.th/en/index.php/facilities/thai-national-observatory-tno/overview}{TNO} 
        telescope, \narit. \\
      & {\bf Tools:} \LaTeX, {\tt latexmk}, PIPT
    \end{tabular}
  }}
\end{tabular}



% \subsection*{Astronomical Observing}
% 
% \begin{itemize}
%  \item 98 nights on 1.9m telescope at \saao, Sutherland\\
%   \hspace*{0.4cm} -- Rapid photometry of CVs (SHOC CCD)\\ %(84 nights) %7
%   \hspace*{0.4cm} -- Polarimetry of magnetic CVs (HIPPO)\\% (Assisting) %7
%   \hspace*{0.4cm} -- Spectroscopy of Dwarf Stars %((SkyMapper follow-up)\\
%  
%  \item 23 nights on 1.0m telescope at \saao, Sutherland \\
%   \hspace*{0.4cm} --  Multi-colour photometry of interacting galaxies \\
%   \hspace*{0.4cm} --  High speed photometry of CVs
%  
%  \item 7 nights on 1.4m Infra-red Survey Facility telescope (IRSF) at \saao, Sutherland \\
%  \hspace*{0.4cm} -- Photometry of interacting galaxies (during ISYA)
%  
% \end{itemize}  
  


\section{Research interests} 

% \vspace{0.2cm}
% \subsection{Astronomy}
% \normalsize
\begin{tabular}{l l}
  \tb \ Astrophysical Accretion on all scales & \tb \ Compact binary stars \\
  \tb \ Astrophysical transients              & \tb \ Multi-wavelength and multi-messenger astronomy \\
  \tb \ Machine Learning                      & \tb \ Bayesian Modelling and Inference \\
%     \item Statistical Modelling
\end{tabular}


\section{Awards and Achievements}
% ---------------------------------------------------------------------------- %
\begin{tabular}{p{0.175\linewidth} p{0.8\linewidth}}
  \itm{2013 -- 2016}     & NRF, \pdp Doctoral Scholarship \\
  \itm{2010 -- 2012}     & South African \ska Postgraduate Scholarship \\
  \itm{2008 -- 2009}     & SKA Undergraduate Bursary Award\\
\end{tabular}

 
\section{Leadership \& Involvements}
% ---------------------------------------------------------------------------- %

\subsection*{Academic}

\begin{tabular}{l l l}
  \itm{2017 -- 2018}  & Postgraduate Mentor                  & Astronomy/Physics Dept. \uct   \\
    \tblblt{3}{Advised undergraduate students on postgraduate opportunities}                  \\
  \itm{2014 -- 2016}  & Postgraduate Student Representative  & Astronomy Dept. \uct           \\
    \tblblt{3}{Mediated student issues within Science Faculty}                                 \\
  \tblblt{3}{Served on Science Postgraduate student council}                                 \\
    \itm{2011}         & Volunteer                            & Siyavula Education             \\
  \tblblt{3}{Translated open source High School science textbooks into Afrikaans}            \\
\end{tabular}

\subsection*{Public Talks}
  \begin{tabular}{l l l}
    \itm{2016 January 23}    & \href{\talksURL/2016Jan_OpenNightSAAO.pdf}{``The Cataclysmic Variables''}  & \saao Open Night      \\
    \itm{2016 June 20}    & \href{\talksURL/2016June_HAC.pdf}{``How the universe creates CVs''}    & Hermanus Astronomy Club    \\
  \end{tabular}


% \newpage
\section{Professional Development}
% ---------------------------------------------------------------------------- %

\subsection*{Conference Attendance}
%
\begin{longtable}{l l l}
%
  \itm{2023 Dec.}   & Magnetism \& Accretion 2023 & \uct Graduate School of Business\\
    \poster{On the discovery of Quasi-Periodic oscillations in the polars J1928-5001 and IGR J14536–5522}{2023Dec-MagnetismAccretion.pdf} \\

  \itm{2017 Nov.}    & IAU Symposium 339: Southern Horizons in Time-Domain Astronomy      & \stias      \\
   \poster{Quasi-Periodic Oscillations in magnetic CVs}{2017Nov_IAUS339.pdf}                  \\
  
  \itm{2017 Nov.}    & .Astronomy9                                          & \saao      \\
  
  \itm{2017 Sept.}    & Deep Learning Indaba                & \wits        \\
%    \poster{Predicting Epileptic Seizures form EEG data with Deep Learning}                  \\
%
  \itm{2016 July}    & \saip Conference                & \uct        \\
    \talk{Quasi-Periodic Oscillations in magnetic CVs}{2016June_SAIP.pdf}                  \\
  
  \itm{2015 Sept}.    & The Golden Age of Cataclysmic Variables and Related Objects -- III    & Palermo, Italy    \\
    \talk{Quasi-Periodic Oscillations in magnetic CVs}{2015Sept_GoldenAge.pdf}                  \\
    
  \itm{2015 June}    & SALT Science Conference 2015              & \stias      \\
    \poster{Probing accretion in magnetic CVs through rapid photometry with SALTICAM}{2015June_SALTScienceConf.pdf}        \\
  
  \itm{2014 July}    & \saip Conference                & \uj        \\
    \talk{Rapid Variability of magnetic Cataclysmic Variable Stars}{2014Aug_SAIP.pdf}                \\
  
  \itm{2013 Sept}.    & The Golden Age of Cataclysmic Variables and Related Objects -- II    & Palermo, Italy    \\
    \talk{Modelling Rapid Variability in Cataclysmic Variable Stars}{2013Sept_GoldenAge.pdf}              \\
  
  \itm{2013 July}    & \ska Joint Radio Transients Conference          & Protea Hotel, Kr\"uger Gate  \\
  
  \itm{2013 Feb}    & StellaNovae                                      & Cape Town  \\
    \poster{Modelling Quasi-Periodic Variability of Dwarf Novae in Outburs}{2013_StellaNova.pdf} \\
    
  \itm{2012 Dec.}    & \ska Postgraduate Bursary Conference            & \stias      \\
    \talk{Modelling Quasi-Periodic Variability in Dwarf Novae during outburst}{2012Dec_SKA.pdf}              \\
  
  \itm{2012 Aug.}    & IAU XXVIII General Assembly              & Beijing, China    \\
    \poster{Modelling Quasi-Periodic Variability in Cataclysmic Variable Stars}{2012Aug_IAU.pdf}            \\
  
  \itm{2011 Dec.}    & \ska Postgraduate Bursary Conference            & \stias      \\
    \poster{A study of DNOs and QPOs in Cataclysmic Variable Stars}{2011Dec_SKA.pdf}                \\
  
  \itm{2011 Mar.}    & Middle-East and African Regional IAU Meeting II (MEARIM-II)      & Cape Town      \\
  %
  \itm{2010 -- 2008 Nov.}  & \ska Postgraduate Bursary Conference            & \stias      \\
  \\
\end{longtable}


\subsection*{Workshop Attendance}
%
\begin{longtable}{l l l}
%
  \itm{2017 Apr.}     & Workshop on Magnetic Accretion            & \saao        \\
    \talk{Observations of Quasi-Periodic Oscillations in magnetic CVs}{2017June_mCVsSAAO.pdf}              \\
  
  \itm{2017 Apr.}     & \ska Big Data Africa Summer School            & Cape Town      \\
   \hack[Led tutorial session]{Outlier detection for time series data}{}                  \\  %TODO: link to notebook
   \hack[Led a student team investigating]{Epileptic seizure prediction from EEG data}{2017Apr_BigDataAfrica.pdf}        \\
  
  \itm{2016 Nov.}     & Workshop on Bayesian Analysis in Physics and Astronomy      & Stellenbosch      \\
    \hack[Led hack project]{Bayesian methods for CCD photometry}{2016Nov_BayesianPhotometryHack.pdf}            \\
  % TODO: JEDI 3: craniometry ....
  
  \itm{2016 May }     & CDS Tools Workshop                & \saao        \\
  \itm{2015 Nov.}     & Workshop on using ALMA for science            & \uct        \\
  \itm{2015 Apr.}     & GPGPU programming workshop              & \uct        \\
  \itm{2014 Nov.}     & \textsc{GADGET} Simulations workshop            & \uct        \\
  \itm{2014 Oct.}     & $2^{nd}$ Machine learning JEDI Workshop          & Cape Town      \\
     \tblblt{3}{Developed classification pipeline for \href{https://www.kaggle.com/c/seizure-prediction}{American Epilepsy Society Seizure Prediction Challenge}}        \\
%      \tblblt{3}{Ranked top 15\% on Kaggle}                \\
  \itm{2012 Feb.}     & IAU International School of Young Astronomers (ISYA)        & \uct, \saao      \\
  \itm{2011 Oct.}     & Workshop on Space Science and Astrophysics           & \acrshort*{chpc}, CSIR, Pretoria   \\
  \itm{2010 Dec.}     & Workshop on Convection in stars            & \uj        \\
  \itm{2010 Jan.}     & \gls*{nassp} Summer School              & \uct \& \saao      \\
\end{longtable}






\vspace{1cm}
\section{Publications}
% ---------------------------------------------------------------------------- %
%TODO: bonus points if you can get the hyperlink to the url embedded in the printed text somehow

\nameusesave
\nameuse{false}
% \nameuserestore

\subsection*{Peer-reviewed}

\begin{refsection}[publications/articlesFA]
  \nocite{*}
  \printbibliography[title={First-Authored},
                     heading=subsubbibliography]
\end{refsection}


\begin{refsection}[publications/articlesCo]
  \nocite{*}
  \printbibliography[title={Co-authored},
                     heading=subsubbibliography]
\end{refsection}


\begin{refsection}[publications/proceedings]
  \nocite{*}
  \printbibliography[title={Conference Proceedings},
                     heading=subbibliography]
\end{refsection}


\begin{refsection}[publications/shortpub]
  \nocite{*}
  \printbibliography[title={Short publications},
                     heading=subbibliography]
\end{refsection}


\section*{Outreach \& Volunteering}
% ---------------------------------------------------------------------------- %
\begin{tabular}{l l l}

  \itm{2024}        & Public Stargazing\\
    \tblblt{3}{Hosted public stargazing event at the Rocklands Highline Festival}      \\

  \itm{2023}        & Public Stargazing\\
    \tblblt{3}{Hosted public stargazing event at the Namaqua Flower Festival}      \\
    
  \itm{2023-2024}    & Metal work & Clean Green Muizenberg  \\
    \tblblt{3}{Community project to make trash bins for the local park from upcycled oil drums}  \\

  \itm{2013 -- 2020}  & Volunteer          & \saao        \\
    \tblblt{3}{Volunteer at \saao Open Night and various public stargazing events}      \\

  \itm{2018}        & &\\
    \tblblt{3}{Hosted public stargazing event at the Festival of Friends, Natures Valley}      \\
  
  \itm{2017-2018}    & Founding Member \& Sport Climbing Coach  & \href{https://www.facebook.com/DreamHigherCT}{DreamHigher} NPO      \\
    \tblblt{3}{Coach Rock Climbing to previously disadvantaged young adults}        \\
  
  \itm{2012 -- 2015}  & Mountain Search and Rescue Member    & \mcsa        \\

\end{tabular}


\section*{Sports \& Culture}
% ---------------------------------------------------------------------------- %

\begin{tabular}{l l l}

  \itm{2022}  &  Project lead        & South African Highline Record             \\
    \tblblt{3}{Established the longest highline in SA at 460m in Mosselbay}              \\
  
  \itm{2011 -- 2015}  &  Committee member      & \uct \msc              \\
    \tblblt{3}{Fulfilled various portfolio roles including Expeditions, Equipment and Ski}            \\

  \itm{2014}    &  Chairperson        & \uct \msc              \\
    \tblblt{3}{Led a committee of 22 people at the head of a society with ~800 members}              \\
    \tblblt{3}{Managed and spent a budget of $\sim$R250 000}                    \\
    \tblblt{3}{Coordinated 5-10 weekly events}                        \\
  
  \itm{2013}    &  Expedition Leader      & \uct \msc              \\
    \tblblt{3}{Led Team on \msc expedition to summit Mt Kenya (5 199m)}                  \\
    \tblblt{3}{Awarded R 14 000 in sponsorship funding}                      \\
  
  \itm{2012}    &  Sports Merit Award      & \uct                 \\
    \tblblt{3}{Led Southern-African team on UIAA Youth expedition to summit the highest mountain in Europe, Mt Elbrus (5 642m)}    \\

  \itm{2011}    &  Sports Performance of the Year Award  & \uct                 \\
    \tblblt{3}{Co-organised mountaineering expedition to summit the Himalayan peak, CB13-A (6 264m)}          \\
    \tblblt{3}{Awarded R 30 000 in sponsorship funding}                      \\
    \tblblt{3}{\href{https://www.news.uct.ac.za/images/userfiles/files/publications/miscellaneous/campussport_2011.pdf}{Article in UCT Campus Sport Magazine}}\\

  \itm{2010 -- present}  &  Waaihoek Leader        & \uct \msc              \\
    \tblblt{3}{Organised and lead numerous international expeditions and multi-day hikes}              \\

\end{tabular}


\section{References}

%  \item Emeritus Distinguished Professor Brian Warner\\Astronomy Department\\University of Cape Town\\Phone: +27 21 650 2391\\email: brian.warner@uct.ac.za

% \begin{center}
  \begin{tabular}{l l}
    
    \reference{Prof. Patrick Woudt}
              {Head of Department Astronomy}
              {\uct}
              {pwoudt@ast.uct.ac.za}
              {+27 21 650 2392}
    \\
    % \reference{Dr. Nadeem Oozeer}
    %           {Data Scientist}
    %           {SARAO}
    %           {nadeem@ska.ac.za}
    %           {+27 21 506 7325}\\
    % \reference{Dr. David Buckley}
    %             {Project Scientist}
    %             {\salt}
    %             {dibnob@saao.ac.za}
    %             {+27 21 447 0025}\\
    \reference{Prof. Bruce Bassett}
              {Head of Cosmology and Machine Learning group}
              {African Institute for Mathematical Sciences}
              {bruce.a.bassett@gmail.com}{}

    \reference{Jason Smythe}
              {Co-founder}
              {Float Capital}
              {jason@float.capital}
  
  \end{tabular}
% \end{center}



 
%  
 
 



\end{document}
